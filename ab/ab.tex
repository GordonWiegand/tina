\documentclass[a4paper]{scrbook}
\usepackage[ngerman]{babel}
%Schrift Quattrocento
\usepackage{quattrocento}
\usepackage{calligra}
\usepackage[T1]{fontenc}
%Schweizer Anführungszeichen
\usepackage[german=swiss]{csquotes}
%Erlaubt Einsatz von Farbe
\usepackage{color}
%UTF 8
\usepackage[utf8]{inputenc}
%Ermöglicht Zierkapitälchen zum Anbsatzanfang
\usepackage{lettrine}
%Fourier Ornamente für Seitenkopf
\usepackage{fourier-orns}
%Ums Seitenkopf zu manipulieren
\usepackage{scrpage2} 
%Kapitelüberschriften in Schriftart des Deckblatts
\addtokomafont{chapter}{\color{red}\rmfamily\Huge\centering\it\calligra}
%Schriftart für grosse Anfangsbuchstaben
\input Zallman.fd
\renewcommand{\LettrineFontHook}{\usefont{U}{Zallman}{xl}{n}}
%Schrift für Überschriften
\def\FontH{\fontsize{10pt}{0mm}\usefont{T1}{frc}{m}{n}}
%Grafiken einbinden
%Abstand zw. Wörtern darf zwecks Blocksatzbildung in Ausnahmefällen bis 1em breit werden.
\setlength{\emergencystretch}{1em}
%bessere Mikrotypographie (margin kerning insb. am Rand)
\usepackage{microtype}
%Seitenzahlen fett
\addtokomafont{pagenumber}{\bfseries}
%To-Do Kommentare einfügen -> \todo{} \missingfigure{}
\usepackage{shapepar}            
%%%%%%%%%%%%%%%%%%%%%%%%%%%%%%%%%%%%%%%%%%%%%%%%%%%%%%%%%%%%%%%%%%%%%%%%%%%%%%%%%%%%%%%%%%%%%%%%%%%%%%
\begin{document}
\pagenumbering{gobble}
%titelseite
\chapter*{Liebe Anna, leiber Birk}
\heartpar{Na vielen Dank! Jetzt sind wir verwirrt. Da denkt man nun, dass man sich erfolgreich in der Wahl seines Freundeskreises auf den vernünftigen Bruchteil der Menschheit beschränkt hat, und dann das: da heiraten welche. Das ist doch schlicht nicht logisch. Steuerliche Vorteile? Geschenkt. Moralische Verpflichtung, weil das erste Kind im Anmarsch ist? Eher auch nicht. Um allabendlich mit einem Stolz der nur aus Traditionsbewusstsein geboren wird, die Hand auf des Sohnes Kopf zu legen, und tief befriedigt zu schnaufen? Gruselige Vorstellung. Also was kann die Erklärung sein, die uns rechnende Menschen zu überzeugen vermag? Na ist nix. Aber --mhh-- wir sind ja jetzt selbst schon neun Jahre verheiratet. Es muss wohl aus Liebe gewesen sein. Da ist kein anderer Grund, als öffentlich und verbindlich zu sagen: {\it wir lieben uns}. Das ist ein bisschen so, wie überall heraus zu posaunen, dass man aufhört zu rauchen. Man verpflichtet sich in aller Öffentlichkeit und akzeptiert eine gewisse Öffentlichkeit seines Beziehungslebens. Aber ehrlich? Das ist manchmal gar nicht das schlechteste. Irgendwie hilft das, die Täler weniger tief, aber die Höhen etwas höher zu machen. Keine Ahnung wieso, aber es funktioniert! Dabei jede Menge Spass und besonders viele Höhen zu haben, wünschen Euch die Wiegands.   
}

\end{document}





