\chapter*{\FontH{\Huge }}
\lettrine[lines=2]{\color{red}A}{ls} die grosse Krähe die Menschen erschuf, lebten diese auf einer Insel mitten Meer. Alle Menschen waren glücklich. Das Meer wimmelte von Fischen, Obst und Gemüse wuchsen prächtig und ab und an liess sich auch ein Wildschwein fangen. Das waren dann immer besondere Tage. Die Menschen zogen ihre schönsten Kleider an und trafen sich auf dem Platz des einzigen Dorfes und machten Musik und sangen solange das Schwein über dem Feuer bruzelte. Krankheiten waren selten und Streit unbekannt, da alles allen gemeinsam gehörte.

Die Menschen achteten die grosse und brachten ihr jede Woche kleine Geschenke. Das waren meistens Muscheln, die die Kinder am Strand gefunden hatten oder der Zahn eines Wildschweins oder auch eine besonders schöne Blume. Dafür beschützte die Krähe die Menschen und zeigte ihnen, wie man das Feld bestellt und Häuser baut.

Karnouk war einer dieser Menschen. Er war jung und schön und wurde von allen geachtet. Er konnte schneller laufen als alle anderen und interessierte sich für alles was die Menschen damals kannten, was allerdings nicht sehr viel war. Am liebsten sass er am Strand und beobachtete die Vögel oder die Fische. Kanouk war der einzige, der wusste, wo sich die Wildschweine am Tag versteckt hielten, aber das verriet er niemanden, denn er dachte, dass die Wildschweine alleine entscheiden sollen, wann sie gefangen werden sollen.

Wie jeden Herbst kamen auch in diesem Jahr die Stürme. Gewöhnlich waren die nicht sehr stark, nur hin und wieder wurden ein paar Palmenwedel weggeblasen, mit denen die Menschen ihre Dächer bedeckten. Dieser eine Sturm war etwas schlimmer und er hielt viel länger an als sonst. Der Wind wollte für fünf Tage und fünf Nächte nicht aufhören zu blasen. Die Menschen verliessen ihre Häuser und suchten Schutz in einer Höhle. Als der Sturm sich endlich wieder gelgt hatte, kehrten sie in ihr Dorf zurück und sahen, was der Wind angerichtet hatte. Viele Dächer waren abgedeckt, aber das war nicht schlimm, das konnte schnell repariert werden. Aber es war noch etwas anderes geschehen. Hier und da sassen kleine Tiere, wie sie die Menschen bis dahin noch nicht gesehen hatten. Und da sie sie vor allem auf den abgerissen Palmenwedeln fanden, nannten sie sie Heuschrecken. 

Alle waren sich einig, dass Heuschrecken hässlich waren. 

\enquote{Das ist kein gutes Zeichen} riefen sie, \enquote{Wir haben die grosse Krähe verägert.} meinten sie. So überlegten alle, was sie wohl zu bedeuten hätten, aber da die Heuschrecken nach ein paar Tagen alle samt von den Vögeln gefressen wurden, gerieten sie bald in Vergessenheit. Nur Karnouk konnte sie nicht vergessen. Ihn quälte die Frage, wo die Heuschrecken wohl hergekommen waren. Hatte sie der Wind geboren? Das konnte er sich nicht vorstellen. Alle Tiere hatten Vater und Mutter, soweit er das beobachten konnte. 

Zehn Wochen zog er sich auf den höchten Berg zurück, und überlegte. Als er zurück kam, rief das Dorf zusammen und sagte: 

\enquote{Meine Freunde, ich habe lange darüber nachgedacht, wo die Heuschrecken wohl hergekommen sind. Ich glaube der Wind hat sie zu uns geweht, denn ihr wisst, sie waren erst nach dem schlimmen Sturm hier. Aber aus dem Wasser kommen sie nicht, denn sie sehen aus, wie unsere Käfer, sie müssen von einer anderen Insel sein.}

Da erhob sich grosses Gelächter bei den Menschen.

\enquote{Von einer anderen Insel?} riefen sie. \enquote{Schau dich doch um. Von unserem berg aus sieht man das ganze Meer und weit und breit ist nur Wasser. Die grosse Krähe hat nur das Meer und unsere Insel erschaffen, dahinter kommt nichts mehr. Und jetzt lass uns in Ruhe mit deinen dummen Ideen}

Aber Kanouk konnte nicht ruhen. Er rief nach grossen Krähe und wollte sie fragen, was das alles zu bedeuten hätte. Gab es doch noch irgendwo eine andere Insel als ihre? Kanouk war davon überzeugt, aber niemand wollte ihm glauben. Gerne wäre er losgeschwommen und hätte gesucht. Aber es stimmte ja was die Leute sagten: es war nichts zu sehen. Und Kanouk konnte keinesfalls ewig schwimmen. Er seufzte und fing wieder an zu überlegen, aber es wollte ihm nichts einfallen. Dann gab ihm die grosse Krähe ein Zeichen. Eine Kokusnuss viel von einer Palme direkt ins Meer, wurde von den Wellen erfasst und trieb hinaus ins Meer. Das war die Lösung! 

Kanouk fing an vieles auszuprobieren, die Leute wunderten sich immer mehr. Er war grosse Steine ins Wasser und bastelte riesige Schalen aus Palmenwedeln, die er auch ins Wasser setzte, bis sie untergingen. Eines Tages rannte er durch das Dorf und rief schreiend \enquote{Ich hab's, ich hab's!}

Verwundert lief das Dorf zu der Stelle am Strand, wo Kanouk in letzter Zeit gesessen und gebastelt hatte. Etwas, dass ein wenig aussah wie eine Schüssel lag mit der Öffnung nach oben im Sand. Sie war aus dem Holz der Palmen gemacht und mit dem Harz der Nadelbäume gestrichen. 

\enquote{Was soll das denn sein?} fragten sie sich und kratzen sich am Kopf. \enquote{Ich nenne es Boot,} sagte Kanouk \enquote{Und ich werde damit über das Meer reisen!}. Lachend kehrten die Menschen in ihre Häuser zurück, nur Kanouk blieb am Strand. Er hatte den Dorfbewohnern zeigen wollen, wie sein Boot fährt, aber niemand wollte zusehen. Es wurde dunkel. Kanouk war voller Wut. Er stiess die Schale ins Meer.
