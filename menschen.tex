\chapter*{\FontH{\Huge }}
\lettrine[lines=2]{\color{red}A}{ls} die grosse Krähe die Menschen erschuf, lebten diese in einem Dorf auf einer Insel mitten Meer. Alle Menschen waren glücklich. Das Meer wimmelte von Fischen, Obst und Gemüse wuchsen prächtig und ab und an liess sich auch ein Wildschwein fangen. Das waren dann immer besondere Tage. Die Menschen zogen ihre schönsten Kleider an und trafen sich auf dem Platz des einzigen Dorfes und machten Musik und sangen solange das Schwein über dem Feuer bruzelte.

Die Menschen achteten die grosse Krähe und brachten ihr jede Woche kleine Geschenke. Das waren meistens Muscheln, die die Kinder am Strand gefunden hatten oder der Zahn eines Wildschweins oder auch eine besonders schöne Blume. Dafür beschützte die Krähe die Menschen und zeigte ihnen, wie man das Feld bestellt und Häuser baut.

Die grosse Krähe lebte auf dem höchsten Berg der Insel auf dem höchsten Baum. Von dort konnte sie die ganze Insel überblicken. Wenn jemand Hilfe brauchte, kam sie geflogen und tröstete die Menschen. Wenn jemand krank wurde, setzte sie sich so lange neben das Krankenbett, bis der Kranke wieder gesund war. Die Menschen liebten die Krähe dafür wie ihre Mütter.

Karnouk war einer der Dorfbewohner. Er war jung und schön und wurde von allen geachtet. Er konnte schneller laufen als alle anderen und interessierte sich für alles was die Menschen damals kannten, was allerdings nicht sehr viel war. Am liebsten sass er am Strand und beobachtete die Vögel oder die Fische. Kanouk war der einzige, der wusste, wo sich die Wildschweine am Tag versteckt hielten, aber das verriet er niemanden, denn er dachte, dass die Wildschweine alleine entscheiden sollen, wann sie gefangen werden wollten.

Einmal brach sich ein alter Mann auf der Suche nach Beeren im Wald ein Bein, als er über eine Wurzel stolperte. Karnouk hörte die Hilferufe und kam herbeigeeilt. Auch die Grosse Krähe kam und sagte, dass Karnouk nun nach Hause gehen könne, sie werde bei dem Mann bleiben. Aber Karnouk hatte eine bessere Idee. Geschickt band er zwei starke Äste so zusammen, dass der alte mann sich darauf legen konnte und Karnouk zog ihn zurück ins Dorf. Dort konnte er von seiner Töchtern versorgt werden, die ihn pflegten, bis das Bein wieder gesund war.

Die grosse Krähe aber rief: 

\enquote{Ihr Menschen, ich habe Euch immer geholfen und war für Euch da. Ich habe Euch die Dinge gegeben, die ihr zum Leben braucht. Alles weitere ist Frevel und erzürnt mich.} Die Menschen wiegten die Köpfe hin her und versprachen, nichts mehr selbst erfinden zu wollen, auch wenn es sehr praktisch wäre. Nur Karnouk schwieg. Er dachte, dass es doch nicht falsch sein kann, den Menschen zu helfen. Aber wusste auch, dass er noch mehr woltle als nur den Menschen zu helfen, er wollte die Welt kennen lernen. Oft sass er am Strand und blickte in die Ferne. Ob es wohl noch mehr Inseln wie seine gäb? Vielleicht sogar noch andere Menschen?

So verging die Zeit, Karnouks Sehnsucht nach der Ferne wurde jeden Tag grösser, aber es war aussichtslos. Überall Meer und wenn er die anderen Dorfbewohner fragte, riefen die nur, er solle sie nur in Ruhe lassen, die Grosse Krähe werde wütend, wenn jemand so rede. Karnouk wollte das nicht glauben und machte sich auf, die grosse Krähe zu besuchen. Es dauerte einen Tag und eine Nacht bis auf den Berg der Krähe zu klettern.

\enquote{Ich weiss, warum Du kommst Karnouk.} sagte die Krähe. \enquote{Du willst hinaus in die Welt, aber ich warne Dich. Das ist gegen meinen Willen, Du sollst ausgestossen werden aus dem Dorf, wenn Du so weitermachst. Ausserdem höre meinen Rat: Dein Tun ist gefährlich. Von hier oben sehe ich die ganze Welt, aber ich sehe nirgends am Horizont auch nur das kleinste Zeichen von Land.} Enttäuscht kehrte Karnouk in sein Dorf zurück. Er hatte seinen Plan aufgegeben.

Wie jeden Herbst kamen auch in diesem Jahr die Stürme. Gewöhnlich waren die nicht sehr stark, nur hin und wieder wurden ein paar Palmenwedel weggeblasen, mit denen die Menschen ihre Dächer bedeckten. Dieser eine Sturm war etwas schlimmer und er hielt viel länger an als sonst. Der Wind wollte für fünf Tage und fünf Nächte nicht aufhören zu blasen. Die Menschen verliessen ihre Häuser und suchten Schutz in einer Höhle. Als der Sturm sich endlich wieder gelegt hatte, kehrten sie in ihr Dorf zurück und sahen, was der Wind angerichtet hatte. Viele Dächer waren abgedeckt, aber das war nicht schlimm, das konnte schnell repariert werden. Aber es war noch etwas anderes geschehen. Hier und da sassen kleine Tiere, wie sie die Menschen bis dahin noch nicht gesehen hatten. Und da sie sie vor allem auf den abgerissen Palmenwedeln fanden, nannten sie sie Heuschrecken. 

Alle waren sich einig, dass Heuschrecken hässlich waren. 

\enquote{Das ist kein gutes Zeichen} riefen sie, \enquote{Wir haben bestimmt die Grosse Krähe verärgert.} meinten sie. So überlegten alle, was sie wohl zu bedeuten hätten, aber da die Heuschrecken nach ein paar Tagen alle samt von den Vögeln gefressen wurden, gerieten sie bald in Vergessenheit. Nur Karnouk konnte sie nicht vergessen. Ihn quälte die Frage, wo die Heuschrecken wohl hergekommen waren. Hatte sie der Wind geboren? Das konnte er sich nicht vorstellen. Alle Tiere hatten Vater und Mutter, soweit er das beobachten konnte. 

Nachdem er sehr lange über die Frage nachgedacht hatte, rief er das Dorf zusammen und verkündete:

\enquote{Meine Freunde, ich habe lange darüber nachgedacht, wo die Heuschrecken wohl hergekommen sind. Ich glaube der Wind hat sie zu uns geweht, denn ihr wisst, sie waren erst nach dem schlimmen Sturm hier. Aber aus dem Wasser kommen sie nicht, denn sie sehen aus, wie unsere Käfer, sie müssen von einer anderen Insel sein.}

Da erhob sich grosses Gelächter bei den Menschen.

\enquote{Von einer anderen Insel?} riefen sie. \enquote{Schau dich doch um. Von unserem Berg aus sieht man das ganze Meer und weit und breit ist nur Wasser. Die grosse Krähe hat nur das Meer und unsere Insel erschaffen, dahinter kommt nichts mehr. Und jetzt lass uns in Ruhe mit deinen dummen Ideen. Du verägerst die Grosse Krähe und dann sind wir alle verloren.}

Karnouk konnte jetzt an nichts anderes mehr denken, als an die Frage, wie er wohl über das Wasser käme. Er war ein guter Schwimmer, gewiss, aber das Meer war viel zu gross. Da half ihm der Zufall. Als er eines Nachmittags am Strand sass, viel gerade vor ihm eine Kokusnuss ins Wasser und trieb mit den Wellen davon. Das war die Lösung. Er musste nur etwas bauen, dass wie eine Kokusnuss ist, aber gross genug, dass er darin Platz hatte!

Kanouk fing an vieles auszuprobieren, die Leute wunderten sich immer mehr und wurden imemr wütender auf ihn. Die Angst vor der grossen Krähe war einfach zu gross. Aber Karnouk hatte keine Zeit sich um solche Sorgen zu kümmern. Er warf grosse Steine ins Wasser und bastelte riesige Schalen aus Palmenwedeln, die er auch ins Wasser setzte, bis sie untergingen. Eines Tages rannte er durch das Dorf und rief schreiend 

\enquote{Ich hab's, ich hab's!}

Verwundert lief das Dorf zu der Stelle am Strand, wo Kanouk in letzter Zeit gesessen und gebastelt hatte. Etwas, dass ein wenig aussah wie eine Schüssel lag mit der Öffnung nach oben im Sand. Sie war aus dem Holz der Palmen gemacht und mit dem Harz der Nadelbäume gestrichen. 

\enquote{Was soll das denn sein?} fragten sie sich und kratzen sich am Kopf. \enquote{Ich nenne es Boot,} sagte Kanouk \enquote{Und ich werde damit über das Meer reisen!}. Da wurden die Dorfbewohner zum ersten Mal richtig neugierig. Es stimmte ja, was Karnouk gesagt hatte. Irgendwo mussten die Heuschrecken hergekommen sein. Da kam die Grosse Krähe geflogen und rief mit donnernder Stimme:

\enquote{Ihr Menschen, ich hatte Euch gewarnt. Ihr sollt nicht Dinge bauen, die ich Euch nicht gezeigt habe. Ich weiss, dass Karnouk dieses Ding alleine gemacht hat. Deswegen will ich nur ihn bestrafen. Karnouk, Du bist ausgeschlossen aus unserer Gemeinschaft und sollst nie wieder in unser Dorf zurückkehren. Und wenn jemand von Euch Menschen auf seiner Seite ist, soll er es jetzt sagen.}

Die Menschen erschraken, senkten die Köpfe und liefen ins Dorf zurück. Kanouk blieb alleine am Strand zurück. jetzt hatte er nichts mehr zu verlieren. Er nahm seine Angel und so viele Lebensmittel, wie er finden konnte, steckte alles in seine Schale und stiess sie hinaus ins Meer. Er sprang hinein und tatsächlich. Die Schale schwamm und er mit ihr. Noch nie im Leben war Kanouk so aufgeregt wie in diesem Augenblick. Die Strömung erfasste ihn und langsam trieb er vom Strand weg in die offene See. Wind wehte ihm durch die Haare. Die Palmen am Strand wurden aus der Ferne immer kleiner. Und so trieb Kanouk davon, um nie wieder auf seine Insel zurückzukehren.

Viele Tage lang sah Kanouk nur Wasser, so weit er blicken konnte. Er bekam Angst. Gegen den Hunger konnte er fischen, trinken konnte er das Wasser des Regens. Als er schon glaubte, nie wieder Land sehen zu werden sah Kanouk eine Möwe. Und da wusste er, dass erbald Land finden würde. So wurde Kanouk der erste Mensch, der das Festland erreichte.

Die Menschen in dem Dorf aber konnten Kanouk nicht vergessen. Was war wohl aus ihm geworden? Und wer half jetzt, wenn sich jemand verletzt hatte. Nur der Trost der Krähe machte niemanden gesund, dass merkten die Menschen jetzt. Und eines Tages vertrieben sie die Krähe und fingen an, gemeinsam eine grosse Schale zu bauen, damit sie hinaus fahren konnte aufs Meer, um Kanouk zu suchen.
