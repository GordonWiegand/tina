\chapter*{\FontH{\Huge Löwenalarm}}
\addcontentsline{toc}{chapter}{Löwenalarm}
\lettrine[lines=3]{\color{red}M}{ama} und Papa schlafen noch. Mama wie üblich etwas lauter als Papa. Ich habe mich zwar noch zu ihnen gelegt, aber das war dann doch langweilig. 

Der aufregendste Ort bei mir zuhause ist die oberste Schublade im Schreibtisch von Mama. Eigentlich darf ich die gar nicht aufmachen, denn dort gibt es auch eine Schere und Reisszwecken und allerhand andere Dinge, die für mich gefährlich sein könnten. Behaupten jedenfalls Mama und Papa und die sagen, dass man mindestens sechs Jahre sein muss, um da aufmachen zu dürfen. Aber ich kann ja spielen, ich sei schon so alt. 

Tausend Dinge gibt es in dieser Schublade, bei den meisten weiss ich gar nicht, wozu die gut sind. Zum Spielen sind sie jedenfalls bestens geeignet. Eine Dose mit kleinen silbernen Dingern geht erst nicht auf, dann purzeln alle auf den Boden. Ein Lineal kommt zum Vorschein, dass kenne ich schon. Eine prima Rutschbahn gibt das Lineal, wenn man es schräg hält. Nach und nach dürfen alle Dinge mal auf die Rutschbahn. Ganz hinten in der Schublade entdecke ich einen Stift, den ich noch nie gesehen habe. Der ist ganz bunt. Aha, wenn man hier drückt, kann man rot schreiben und wenn man hier drückt blau. Gelb und schwarz gibt es auch und rosa, meine Lieblingsfarbe, ist auch dabei.

Feen sind ja meistens rosa, also fange ich an eine Fee zu malen. Papier gibt es hier genug. Erst eine grosse Fee, dann noch eine und dann noch drei kleine dazu. Eine Feenfamilie mit zwei Mamas. Aus dem Schlafzimmer höre ich meine Mama schnarchen. Wie ein Löwe klingt das, also male ich auch noch einen Löwen. Mit grosser Mähne und aufgerissenem Mund, der brüllt nämlich gerade.

Und jetzt noch einen Knochen für den Löwen zum Fressen, der soll blau werden, aber irgendwie will der Stift nicht mehr blau malen. Also ab ins Kinderzimmer einen blauen Stift suchen.

Als ich endlich einen gefunden habe und zurückkomme sind die Feen weg! Verschwunden! Nicht mehr da! Die Papierblätter liegen noch genau da wo ich sie gelassen habe, der Löwe ist auch noch da, aber die Feen sind weg.

Ich suche sie in der ganzen Wohnung und finde sie tatsächlich. Sie haben sich hinter dem WC versteckt und haben sich ganz fest aneinander geklammert.

\enquote{Hilfe!}, rufen sie, \enquote{Ein riesiger Löwe war da eben mit offenem Mund, der wollte uns fressen!}

Oh je. Jetzt haben die Feen Angst vor dem Löwen bekommen. Das habe ich nicht gewollt, da habe ich gar nicht daran gedacht. Ist ja aber eigentlich klar! Ich versuche die Feen zu beruhigen, aber es hilft nichts. Immer wieder rufen sie \enquote{Wir haben Angst, wir haben Angst!}

Dann habe ich eine gute Idee. Ich nehme den blauen Stift und male noch einen Käfig um den Löwen herum. Da kommt der nicht raus, wie im Zoo. Ausserdem noch einen Apfel um zu zeigen, dass das ein Löwe ist, der lieber Äpfel als Feen ist. Gerade will ich das Bild den Feen wieder zeigen, da sehe ich, dass alle Feen wieder zurück gekommen sind und wieder auf dem Papier sind, auf das ich sie gemalt habe. Da bin ich beruhigt, alles ist wieder gut.

Unterdessen sind auch Mama und Papa aufgestanden. Ich erzähle ihnen, was passiert ist, aber Papa, der noch ganz verschlafen ist und sich die Augen reibt, antwortet nur \enquote{Toll mein Schatz, wie du die Feen und den Löwen gemalt hast.} Der hat mich wohl gar nicht richtig verstanden. Eltern sind manchmal sonderbar, die aufregendsten Dinge im Leben verpassen die immer.  \hfill {\color{red}\decofourleft}
