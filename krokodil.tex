\chapter*{\FontH{\Huge Aus dem Leben des Karl Rebosam}}
\addcontentsline{toc}{chapter}{Aus dem Leben des Karl Rebosam}



Kleines Krokodil, Du siehst ja so traurig aus?

Ich glaube, das bin ich auch.

Aber warum denn kleines Krokodil, sieh doch mal, wie schön die Sonne scheint.

Die Sonne scheint heute nicht, sie blendet.

Und die Blumen, riech nur wie sie duften!

Die Blumen duften nicht, sie stinken.

Aber dein Fluss, er ist so herrlich klar heute!

Ja, damit mich alle anstarren können und sich über mich lustig machen.

Aber wer macht sich denn über die lustig kleines Krokodil?

Na einfach alle.

Ich zum Beispiel ja nicht. Komm her zu mir, kleines Krokodil und lass dich einmal ganz fest in den Arm nehmen.

Das stimmt, es ist schon wieder etwas besser. Ich habe aber trotzdem Lust, einfach los zu weinen.

Dann weine doch einfach kleines Krokodil. Leg dich hier auf meinen Schoss und weine. Ich werde dich so lange etwas streicheln, wenn es dir nichts ausmacht.



Sieh mal, deine grossen Krokodilstränen bilden einen kleinen Bach. Er schlengelt sich schon in seinem eigenen Bett zum Fluss. Schau nur, das lockt die Fische an, sieh nur wie sie im Wasser springen. Und da kommen auch die Vögel, die wollen wissen, was die Fische machen. Und von den Vögeln werden all die anderen Tiere angelockt. Sieh nur kleines Krokodil, alle Tiere sind jetzt hier und geben dir einen Kuss.

Aber da ist ja gar kein anderes Tier?

Oh, du darfst nicht mit den Augen sehen, die sind ja noch voller Krokodilstränen. Du musst die Augen schliessen, dann siehst du sie.


