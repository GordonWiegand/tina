\chapter*{\FontH{\Huge Jonathan}}
\lettrine[lines=2]{\color{red}K}{rrrawum.} Mit einem lauten krachen fällt die Rüstung von Ritter Odo von Gladez um. Die weisse Frau von Burg Lauenstein kommt durch die Wand gleich hinter der Rüstung geflogen und sieht sich den Haufen alten Blechs auf dem Boden an.

\enquote{Jonathan!} ruft sie, \enquote{Komm sofort hierher!} Mit gesenktem kommt auch Jonathan. Jonathan ist der Sohn der Weissen Frau, die eigentlich Otilia Brigitta Walburg Peternella von Abendberg heisst. Aber alle nennen sie die weisse Frau und das schon seit vierhundert Jahren. Sie ist nämlich ein Gespenst und die Mama von Jonathan, der natürlich auch eines ist. 

\enquote{Was soll ich nur mit Dir machen?} seufzt die Mama. Andauernd wirft Jonathan etwas um. Er passt einfach nicht auf, wenn er zu schnell durch die Wände geflogen kommt und nicht merkt, dass in dem Raum zum Beispiel eine Vase oder Kleiderständer steht. Gespenster können nämlich nur durch Wände fliegen, durch andere Sachen nicht. Und Jonathan war eben ein besonders schreckhaftes Gespenst. Immer wenn er glaubt, dass Menschen in der Nähe sind, nimmt er Reissaus und saust durch Burg Lauenstein, um sich zu verstecken. Besonders vor Kindern hat er furchtbare Angst. Die sind immer laut und frech und ärgern sich gegenseitig. 

Burg Lauenstein müsst ihr wissen, ist eine alte Burg. Tagsüber kommen Besucher aus der Umgebung und sehen sich an, wie die Menschen früher so gewohnt haben. Und zu einer alten Burg gehören eben auch Gespenster, die man allerdings nie am tage sieht und auch Nachts nur, wenn man sehr viel Glück hat. Seitdem irgendjemand mal auf die Idee gekommen ist, zu behaupten, dass Gespenster ganz gefährlich sind, haben sie angefangen, sich in den Kellern und Türmen oder wenigstens den Truhen der alten Burgen und Schlösser zu verstecken. Die allermeisten haben sich aber in Büchern versteckt und leben jetzt nur als Geschichte weiter. Eigentlich kenne ich kein Gespenst mehr, dass nicht nur in Büchern lebt, die weisse Frau und ihr Sohn Jonathan sind vielleicht die letzten.


