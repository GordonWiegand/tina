\chapter*{\FontH{\Huge Das schwarze Licht}}
\addcontentsline{toc}{chapter}{Das schwarze Licht}
\lettrine[lines=3]{\color{red}K}{rokusse} und Adonisröschen gaben dem Wald gerade die ersten Farbtupfer nach einem strengen  und langen Winter als ein Kurier im Dorf eintraf und die schreckliche Nachricht überbrachte. Ein Guhl war aus seinem Schlaf erwacht. Guhle sind mächtige Wesen, die tausend Jahre schlafen. Ein unvorsichtiger Jäger hatte seinen Pfeil versehentlich in die Höhle des Guhl geschossen und ihn damit geweckt.

Der Guhl war in seiner Wut losgezogen und verbreitete überall schwarzes Licht. Es gibt nichts dunkleres auf dieser Welt, als schwarzes Licht. Egal wie gross das Feuer ist, dass man anzündet, das schwarze Licht des Guhls ist stärker. Das schwarze Licht kann alles durchdringen und wo es ist, ist alles schwarz. Nicht dunkel wie in der Nacht, nein, alles ist schwarz. Es ist unmöglich etwas zu erkennen. Und wo das schwarze Licht ist, hören die Pflanzen auf zu wachsen und die Tiere sterben. Bald gibt es keine Nahrung mehr und die Menschen müssen in der Dunkelheit verhungern.

Noch bevor die Menschen des Dorfes einen Plan fassen konnten war der Guhl da und mit ihm das schwarze Licht. Schwärze legte sich über das Dorf und kroch durch jede Ritze. Zündete man eine Kerze an, verbrannte man sich nur die Hände, denn auch die Flamme war schwarz. Mühsam tasteten sich die Menschen durch ihre Häuser auf der Suche nach Essen. Aber die Vorräte waren nach dem Winter aufgebraucht und etwas neues wuchs nicht.

Schnell machte sich grosse Angst breit und die Menschen verbarrikadieren sich in ihren Wohnungen. Nur ein Mädchen hatte keine Angst: Onta. Onta war schon blind geboren worden, ihre fehlte das Licht also nicht. Sie wusste genau wie viele Schritte es vom Tisch bis zur Tür und von dort bis überall hin im Dorf sind. Sie konnte sogar im Wald rennen, so genau hatte sie sich eingeprägt, wo welcher Baum stand.

Und so machte sich Onta auf den Weg, den Guhl zu suchen, um ihn zu besänftigen. Wie das gelingen könnte, wusste sie selbst noch nicht so genau. 

Noch nie hatte sie den Wald so leise erlebt. Um so weiter sie kam, um so vorsichtiger musste sie sich den Weg ertasten. Aber sie wusste, dass sie auf dem richtigen Weg war, denn der Gestank des Guhl wurde von Stunde zu Stunde stärker. Nach ungefähr einem Tag Wanderung war sie sicher am Ziel zu sein. Der Guhl musste jetzt ganz in der Nähe sein.

Onta setzte sich auf einen Stein und wartete. Um sich die Zeit zu vertrieben und gegen die Angst und die Stille nahm sie ihre kleine geschnitzte Flöte und spielte ein Lied. Plötzlich stand der Guhl hinter ihr. Onta konnte ihn spüren. Er musste riesig sein. Niemand wusste genau, wie so ein Guhl aussieht, vielleicht ein bisschen wie ein riesiger Bär. Jedenfalls brummte er wie einer, blos sehr viel lauter. Aber Onta merkte, dass nichts Böses in diesem Brummen lag. 

Den Geräuschen nach hatte sich der Guhl auf den Boden gesetzt. Onta spielte einfach weiter und immer weiter. Das Brummen des Guhl war immer gleichmässiger geworden und einem Schnarchen gewichen. Der Guhl war wieder eingeschlafen. Onta spielte weiter so lange sie konnte. Es könnte ein Tag gewesen sein oder zwei, als sie das leise Knacken von Zweigen hinter sich hörte.

Mit ganz leiser Stimme flüsterte eine Stimme ihren Namen. Da hörte Onta auf zu spielen, der Guhl schlief weiter. Die Stimme gehörte einer Jägerin des Dorfes, die sie suchen gekommen war. Als der Guhl eingeschlafen war, war auch das schwarze Licht verschwunden. Onta hatte dies noch nicht bemerkt, da bis hier noch keine Vögel zurückgekehrt waren.

Die Menschen des Dorfes bauten eine grosse Mauer um den Guhl. Nicht um ihn einzusperren, das wäre unmöglich gewesen, sondern um seinen Schlaf zu schützen.  \hfill {\color{red}\decofourleft}
