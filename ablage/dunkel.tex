\chapter*{\FontH{\Huge Der dunkle Weg}}
\addcontentsline{toc}{chapter}{Der dunkle Weg}
\section*{Die Räuber}
\lettrine[lines=3]{\color{red}S}{päter} erzählte man sich, dass der Tag an dem Onta geboren wurde, schon ein Zeichen war. Der Herbst war kalt gewesen, dichte Nebel zogen durch die Wälder, die sich auch am Tage nicht auflösten. In der Nacht zur Wintersonnenwende verschwand der Nebel und der erste Schnee bedeckte die Bäume. In dieser Nacht kam Onta zur Welt.

Ihre Mutter starb bei der Geburt und der Vater war unbekannt. So wuchs Onta bei den heiligen Frauen auf. Die heiligen Frauen wachten darüber, dass die Menschen die Gebote der Göttin Freyja ehrten und einhielten. Sie lebten zurückgezogen in den Wäldern und lebten von den Almosen der Bauern. Die Wintersonnenwende nannten die Menschen Jul und verehrten Freyja. Deswegen glaubten sie, dass Onta wohl zu einem Leben für die Göttin bestimmt war. Aber sie irrten sich.

Die heilgen Frauen lehrten sie, wie man im Wald wochenlang auf sich alleine gestellt überleben kann. Wie man Fallen stellt, um Hasen zu fangen und wie man sich einen Unterschlupf für den Winter bauen kann. Die lehre nsie auch lesen und schreiben.

Nach zwei Mal sieben Jahren war Onta erwachsen. Die heiligen Frauen sagten, dass es Zeit wäre, sie zur Priesterin zu weihen. Sie bereiteten alles für die grosse Zeremonie vor. Die Abuern der Umgebung brachten die köstlichsten Speisen, denn so war es Brauch. Davon wurden aber die Räuber angelockt, von denen es in dieser Zeit sehr viele gab.

Als die heiligen Frauen schliefen überfielen sie ihr Lager. In ihrer Angst rannten die heiligen Frauen davon und versteckten sich im Wald. Nur Onta blieb. Sie stellte sich in die Mitte des Lagers und wartete. Mit lauten Schreien stürzten die Räuber hinter den Hecken und Bäumen hervor. Sie hatten sich das Gesicht mit Russ schwarz bemalt. Auf ihren Lederjacken glänzten rote Flecken im Mondlicht. Sie schwangen die Äxter und Schwerter über ihren Köpfen und rannten brüllend auf die Hütten und Zelte zu.

Beim Anblick der jungen Frau blieben sie stehen. Noch nie hatte sich ihnen jemand in den Weg gestellt, sie mussten nur sehr selten Kämpfen, weil die Menschen vor ihnen Angst hatten und flohen, bevor sie da waren. und da stand plötzlich Onta.

Nachdem sich die Räuber vergewissert hatten, dass  es keine Falle war, fingen sie an zu lachen und stellten sich im Kreis um Onta. Langsam kamen sie immer näher, aber Onta stand bewegungslos an der selben Stelle. Der Anführer der Räuber, ein Kerl mit langem verfilzten Bart, der noch dazu erbärmlich stank kam auf sie zu. Noch bevor er irgendetwas sagen konnte, traf ihn Ontas Blick. 

Onta war ein Mensch, der völlig frei von jeder Angst war. Sie fürchtete sich vor nichts. Mut zu haben, bedeutet, seine Angst überwinden zu können. Mut hatte Onta nicht. Den brauchte sie auch nicht. Der Räuberhauptmann begriff das sofort. Es verunsichert ihn und je länger er in ihre klaren blauen Augen blickte, die selbst in der NAcht gut zu erkennen waren, desto mehr verunsichert war er. Bisher hatten alle Menschen vor ihm gezittert und die Frauen noch am meisten. Aber nicht diese hier.

Auf seine Frage wer sie sei und was sie wolle erhielt er keine Antwort. Onta blickte ihm nur in die Augen. Der Räuberhauptmann hatte mittlerweile Angst bekommen. Diesen Blick konnte er nicht ertragen. Und Onta blickte nicht weg, keinen Moment. Der Hauptmann brüllte sie an, hielt ihr einen Dolch unter das Kinn, drohte ihr. Aber genauso hätte er einen Baum anschreien können. 

Er liess den Dolch sinken und wendete seinen Blick ab. Er hatte keine Kraft mehr. Er rief seine Männer und befahl ihnen wieder in den Wald zu gehen. \itshape{Bist Du ein Geist? Bist Du der Geist meiner Mutter? Bist Du der Geist der Mütter aller der Menschen, die ich als Räuber getötet habe? Antworte mir!} und als Onta weiter nur regungslos dastand setzte er hinzu \itshape{Ich bitte Dich. Sag mir wer Du bist!}

Er setzte sich auf den Boden, genau vor Onta und begann zu erzählen.

\enquote{Mein Name ist Elof. Ich bin Anführer der Elof-Bande. Alle Menschen in diesen Wäldern fürchten sich vor mir und den meinen. Als Kind wurde das Dorf in dem ich geboren wurde von Räubern überfallen. Keiner hat überlebt. Mich haben sie mitgenommen und zu einem der ihren gemacht. Heute bin ich ihr Anführer. Ich habe selbst viele Dörfer überfallen und Menschen getötet. Ich hatte Spass daran und es hat mich reich gemacht. Aber in dieser NAcht ist etwas in mir zerbrochen. Ich wünschte ich wäre wieder der kleine Junge und dürfte ein normales Leben in einem netten Dorf führen. }

Dann schwieg er eine Weile und fragte wieder

\enquote{Wer bist Du und was hast Du mit mir gemacht?} 

Aber er erwartete keine Antwort mehr. Er stand auf und ging langsam und schweigen zu seinen Männern zurück.

Kurz bevor er den Wald erreichte, rief Onta mit lauter Stimme, so dass es auch die Männer im Wald hören konnten:

\enquote{Mein Name ist Onta.} Und damit war ihr Mythos geboren.

\section*{Die Reise beginnt}
Was Onta da getan hatte, wusste sie selbst nicht. In der Zeit als Onta lebte, glaubten die Menschen an Zauberei und so glaubten sie auch, dass Onta eine Zauberin sein müsse. Die heiligen Frauen war es nicht mehr geheuer, so eine wie sie in ihrem Bund zu haben und sie verweigerten ihr die Priesterinnenwürde und schickten sie fort.





  \hfill {\color{red}\decofourleft}
