\chapter*{\FontH{\Huge Erdbeeren}}
\addcontentsline{toc}{chapter}{Erdbeeren}
\begin{mdframed}[style=mystyle]
\lettrine[lines=3]{\color{red}S}{ommer.} 1 und 2 besuchen ihren Opa. Die Sonne scheint und die neuen Bikinis warten darauf, endlich am Baggersee eingeweiht zu werden. Noch schnell eine Träne verstecken, als sich Mama und Papa verabschieden, die wollen noch weiter, Freunde besuchen. 1 und 2 dürfen zum ersten Mal alleine Ferien machen beim Opa. Der wohnt jetzt nämlich noch gar nicht so lange in dem haus, der ist erst eingezogen.

Es ist schnell entschieden, dass sie im Zelt im Garten schlafen wollen. Opa hat einen prima Garten. Apfel- und Zwetschgenbäume stehen da, Sauerkirschen gibt es, die die beiden aber nicht so mögen, Johannesbeeren, Stachelbeeren und natürlich die Erdbeeren. Die sind die besten, da sind sich beide einig. Frisch pflücken, klein drücken, Milch und ein bisschen Zucker dazu und dann so viel davon essen, bis man Bauchschmerzen bekommt. Traumhaft.

Und genau neben diesem Erdbeerbeet steht ihr Zelt. Opa hat es schon aufgebaut, jetzt muss es noch eingerichtet werden. Ganz oben wird eine Taschenlampe aufgehängt, an den Seiten werden die wichtigsten Dinge verteilt: Plüschtiere, ein Kompass, falls man eine Schatzkarte findet und den Schatz suchen muss, Buntstifte, obwohl man im Zelt eigentlich nicht so gut malen kann, aber die sind ganz neu, die müssen in der Nähe sein. Die Urkunde aus den letzten Skiferien, ein Foto von Mama und Papa. Für den aufblasbaren Delphin ist leider kein Platz mehr, der muss draussen bleiben. Und zwischen all den Dingen ihre Schlafsäcke. Papa nennt die Schlaftüten, was beide aber kindisch finden.

Am Abend hat Opa zur Begrüssung den Grill angemacht, es gibt Bratwürste und Kartoffelsalat. Aber wirklich etwas zu essen schaffen die beiden nicht mehr. Bauchschmerzen von rosafarbener Erdbeermilch. Dann sitzen alle noch lange draussen und erzählen, was in letzter zeit so passiert ist. In der Schule macht es beiden Spass, aber Ferien sind schöner. 1 nervt vor allem ein Mädchen aus ihrer Klasse, die ist einfach nur blöd. 2 ist gerade erst in die Schule gekommen, die nervt noch gar nichts. Das kommt noch, weiss 1. Lebenserfahrung hat sie jetzt schon genug, das beurteilen zu können.

Mittlerweile haben die vielen Vögel, die es hier gibt und den ganzen tag ein risiges Konzert veranstalten auch aufgehört zu zwitschern. Eine herrliche Ruhe liegt im Garten. So etwas sind die beiden, die in der Stadt wohnen, nicht gewohnt. Dort machen die ganze Nacht Autos und Strassenbahnen und sonstige Dinge einen riesen Lärm. Still es es da nie. Irgendwann merken die beiden Schwestern, als sie gerade diskutieren, ob sie schon zu alt sind für Biene Maya, dass Opa die Augen zugefallen sind. Sie müssen kichern. Sie räumen noch schnell das letzte Geschirr in die Küche, wovon Opa wieder wach wird. Aber nur so viel, dass er sich gerade noch verabschieden kann und sich schlafen legt.

Die beiden liegen noch lange wach und gucken in den klaren Sternenhimmel. Sogar Glühwürmchen gibt es hier. 2 versucht eins zu fangen, das ist gar nicht so leicht. Plötzlich sieht sie ein viel grösseres Glühwürmchen mitten im Erdbeerbeet. Aber Moment, dass ist ja gar kein Glühwürmchen. Das leuchtet ja rot! Die Schwestern gucken erstaunt. Was da leuchtet, ist eine Erdbeere.

Vorsichtig pflücken sie die Erdbeere. Seltsam. Mit ihrem Taschenmesser schneiden sie die Erdbeere vorsichtig auseinander. Auch innen leuchtet sie.

\enquote{Wetten, dass du dich nicht traust, so eine Hälfte zu essen.} stichelt 1. 

2 ist beleidigt. Natürlich traut sie sich das. Eigentlich traut sie sich immer alles, was 1 sagt. Das weiss 1 und nutzt das gelegentlich aus, 2 zu Sachen anzustacheln, die ihnen eigentlich verboten sind. Oder verboten wären, wüssten Mama und Papa davon. Das kann 2 nun mal schon besser abschätzen. Zack ist die Hälfte der Erdbeere in 1 Mund verschwunden.

\enquote{Und jetzt du die andere Hälfte.} 2 weigert sich erst so ein bisschen. Womöglich ist die Erdbeere ja schon schlecht und damit giftig. Es kommt zur üblichen Schwesterndiskussionen, die immer nach genau demselben Muster abläuft. Es geht mit dem Vorwurf Feigling los, wird mit selber erwiedert und dann geht das wie beim Tischtennis eine Weile so hin und her. 1 ist aber diesmal klar im Vorteil, denn sie hat die eine Hälfte ja schon gegessen. 2 muss auch, wil lsie nicht den Rest der Ferien das überlegene Gesicht der kleinen Schwester sehen. Das schlimmste, was passieren kann, sind Bauchschmerzen, überlegt sie. Die sind nach einem Tag aber sicher vorbei, es gibt ja schliesslich Medikamente. Also gut, und auch die zweite Hälfte ist gegessen.

Eine Weile warten beide, ob sie Bauchschmerzen bekommen, es passiert aber nichts und beim Warten schlafen sie ein.

\begin{center}
\aldineleft
\end{center}

Der nächste Morgen beginnt zunächst ganz normal. 1 hatte dann doch ein bisschen Angst bekommen, so im Zelt, deswegen war 2 mit ihr dann doch ins Haus gegangen, um weiterzuschlafen. Opa hatte schon damit gerechnet und ein Bett vorbereitet. Mal sehen, ob es nächste Nacht klappt. Opa geht noch schnell zum Becker, frische Brötchen kaufen. Die beiden Schwestern sitzen draussen auf der Veranda und trinken heissen Kakao. Die Vögel zwitschern wieder, als gäbe es einen Wettbewerb, wer das wohl am lautesten kann. Die Sonne scheint auch schon wieder mit ganzer Kraft, 2 muss blinzeln und schliesst die Augen.

\enquote{\dots und dann sagt der freche Sperling zu mir, er hätte den Wurm zuerst gesehen. Dass muss man sich einmal vorstellen. Sperlinge sind doch weiss Gott die frechesten Vögel weit und breit.}

\enquote{Ja, ja, da sagen sie ein wahres Wort, Frau Amsel, erst neulich habe ich \dots}

2 erschrickt. \enquote{Hast du das auch gehört?} fragt sie 1. Die schlürft nur weiter an ihrem Kakao und überlegt, ob man den auch mit Erdbeeren kombinieren kann.

\enquote{Was meinst Du?}

Aber auch 2 hört es nicht mehr. Sie konzentriert sich ganz fest, aber nichts. Um noch besser zu hören schliesst sie wieder die Augen und da sind die Stimmen wieder:

\enquote{Und überhaupt, was für ein protziges Nest sich diese Meise gebaut hat. Geschieht ihr ganz Recht, dass sie einen Kuckuck gross ziehen musst.}  

Augen wieder auf, Stimmen weg. Augen zu, Stimmen da. Auch 1 muss das ausprobieren und auch bei ihr funktioniert das. Beide sind aufgeregt. Sie verstehen die Stimmen der Vögel! Unglaublich. Sie schliessen die Augen und hören zu.

\enquote{Und haben sie gesehen meine Liebe. Die Erdbeere, die die alte Hexe gepflanzt hat ist auch weg.}

\enquote{Die Zauberbeere? Nein! Wer mag die wohl gegessen haben! Hoffentlich nicht die Hexe selber. Haben sie die übrigens gesehen in letzter Zeit?}

\enquote{Na ich glaube die hat den Wald seit Wochen nicht mehr verlassen. Seit sie Herrn Rabe gefangen hat, habe ich nichts von ihr gehört. Wie es dem wohl geht? Vielleicht hat der Mann, der jetzt hier wohnt, die Erdbeere gegessen. Oder diese beiden jungen Mädchen, die zu Besuch sind. Schauen sie nur, da unten sitzen sie und lassen sich die Sonne auf den Bauch scheinen. Versteh einer die Menschen. Sitzen nur rum und suchen nie Futter. Also nein, also nein.}

1 und 2 sehen sich jetzt an. Das ist ja unglaublich. Die leuchtende Erdbeere wurde von einer Hexe gepflanzt? Was mag das für eine Hexe sein? Und warum hat sie das getan? Tausend und eine Frage stellen sich die Schwestern gegenseitig, ohne sie beantworten zu können.

Als der Opa zurückkommt und schon von weitem mit der grossen Papiertüte mit den frischen Brötchen winkt, ist beiden klar, dass sie erst einmal das Geheimnis für sich behalten wollen. Mit vollem Mund fragt 2:

\enquote{Du, Opa, gibt es hier eine Hexe?}

Opa muss lachen.

\enquote{Ihr habt mit dem Nachbarn gesprochen, stimmt's? Die alte Frau, der vorher das Haus hier gehört hat, mag zwar ein bisschen eigentümlich sein, aber eine Hexe ist sie sicher nicht. Das behaupten die Leute nur. Und das ist überhaupt nicht nett, wenn ihr mich fragt.}

\enquote{Was ist mit der Hexe, ähm, alten Frau jetzt?}

\enquote{Ach, die hat noch ein kleines Hüttchen oben am Waldrand. Das Haus hier war ihr zu gross, da ist sie dahin gezogen. Aber ich weiss gar nicht, ich habe sie schon lange nicht mehr gesehen. Aber wenn ihr wollt, können wir sie besuchen, ich wollte so wieso ein bisschen mit euch wandern gehen heute, da können wir ja dort vorbei.}

Die Schwestern sind einverstanden. Sie sind zwar schon ein bisschen ängstlich, immerhin wissen sie sicher, dass es eine richtige Hexe sein muss, wenn sie so Erdbeeren pflanzen kann, aber die Neugier ist einfach grösser. Noch grösser ist allerdings die Hitze, weswegen die Wanderung auf morgen vertagt wird. Heute steht erst einmal baden an. Ab in den Fluss heisst das. Nach vielen Jahren, die er verschmutzt gewesen ist, ist der Fluss jetzt wieder so sauber, dass man bedenkenlos in ihm baden kann.

Auf dem Weg zum Fluss kommen die drei am Wildpark vorbei. Es gibt Wildschweine zu sehen und ein grosses Gehege mit Rehen und Hirschen. Sogar ein weisses Reh ist dabei. Albino nennt man das, erklärt Opa und dass das Reh allerdings sehr krank sein, keiner wisse warum.

\enquote{ Der Förster sagt, dass es wohl bald sterben könnte. Da habt ihr es wenigstens nochmals gesehen.} meint er.

Die ganze Zeit hören sie immer wieder den Vögeln zu. Unglaublich, wie viel die zu reden haben. Wer hätte gedacht, dass das viele Gezwitscher, das man immer hört, eigentich vor allem Tratsch und Klatsch ist. Wer wo jetzt neu ein Nest gebaut hat, wird besprochen. Bei welchem haus jetzt eine Katze wohnt. Wie schmutzig die Tauben angeblich sind und solche Dinge eben. Am schlimmsten sind die Enten. Ständig sagen sie \enquote{Guten Tag} zueinander. So als ob sie sich das erste Mal sehen. Kaum dreht sich eine Ente um und wieder zurück, schon heisst es wieder \enquote{Guten Tag, mein Herr} oder \enquote{Guten Tag, meine Dame.} Vielmerh eigentlich nicht. Manchmal noch ein paar Floskeln zum Wetter. Überhaupt sind Vögel alle immer ausgesucht höflich. 1 und 2 ahmen das die ganze Zeit nach, was den Opa dann doch ein wenig verwundert.

\begin{center}
\aldineleft
\end{center}

Am nächsten Morgen ist es zwar wieder so heiss, die beiden Schwestern wollen aber trotzdem unbedingt zu der alten Frau. Opa winkt ab.

\enquote{Nein meine Lieben, geht ihr gerne alleine, mir ist es zu heiss zum Wandern.} Es wird noch kurz erklärt, wo es lang geht, und dann los.

Beide Mädchen sind schon etwas aufgeregt und ängstlich.

\enquote{Und was ist, wen nsie uns wie bei Hänsel und Gretel\dots?} fängt 1 an zu fragen, wird aber von ihrer Schwester unterbrochen.

\enquote{Paperlapapp.} Sie kann zwar auch nicht sagen, auf was sie sich einlässt, aber ihr Gefühl sagt ihr, dass es schon nicht wirklich gefährlich werden kann. Sie erreichen den Wald, der hier sehr wild und dicht ist. Nach der vielen Sonne auf den Wiesen auf dem Weg, wirkt de Wald richtig düster, aber eigentlich nicht bedrohlich.

Als die Hexe dann völlig unerwartet neben ihnen steht, sind beide sehr erschrocken. In der Hand hält sie einen Bund irgendwelcher Kräuter und trägt eine schmutzige braune Kittelschürze.

\enquote{Was machen denn zwei so hübsche junge Mädchen bei dem Wetter hier im Wald. Ihr seid ja gar nicht baden?}

Keine der Schwestern weiss etwas zu sagen.

2 hat mehr Mut und denkt sich, wenn das wirklich eine Hexe ist, sollte man vielleicht nichts verheimlichen. Eine echte Hexe merkt bestimmt, wenn man schwindelt.

\enquote{Unser Opa ist der Mann, der ihr Haus gekauft hat. Und er hat uns gesagt, wo sie wohnen und da dachten wir, dass wir sie vielleicht einmal besuchen sollten.}

Die Alte brummt etwas vor sich hin, dann stutzt sie, überlegt und mustert die beiden von oben bis unten.

\enquote{Eine von Euch hat die Erdbeere gegessen, richtig? Und jetzt könnt ihr die Vögel verstehen. Ja, die Erdbeere hatte ich ganz vergessen, das passiert im Alter.}

\enquote{Beide.} geben die Schwestern zu, denn irgendwie haben sie nicht das Gefühl, dass die Frau böse ist. 

\enquote{Na was sagen sie denn, die Vögel? Hört es euch gut an, spätestens heute Abend ist das wieder vorbei. Aber es wird auch wieder eine neue leuchtende Erdbeere wachsen, nächste Woche oder übernächste.} 

Die beiden erzählen, was sie so gehört haben, auch, dass sie, die Hexe, einen Raben gefangen haben soll. Die Hexe lacht und sagt, sie sollen mal mitkommen, sie können ja selber hören, was der Rabe zu erzählen hat. Und so gelangen sie zu dem Haus, in dem die hexe wohnt. Ein einfaches Hüttchen, eine Art Gartenhäuschen. hier ist es bestimmt ziemlich kalt im Winter, denkt 2. 

Der Rabe sitzt auf dem Rahmen einer Hollywood-Schaukel und hat einen Verband um den linken Flügel. Die Mädchen schlissen die Augen und hören den Raben sagen, dass er sich den Flügel gebrochen hat, den er aber lustigerweise Arm nennt und das die Hexe ihm geholfen haben.

Beide sind beruhigt und kein bisschen ängstlich mehr. Es gibt kalten Kräutertee, der sehr gut schmeckt. 

\enquote{Und haben die Vögel ein weisses Reh erwähnt? Deswegen habe ich die Erdbeeren überhaupt nur gepflanzt. Ich spüre, dass es diesem Reh nicht gut geht, es ist irgendwo eingesperrt oder gefangen. Und ich dachte, wenn ich den Vögeln zuhören kann, die aus der Luft ja alles sehen, erfahre ich vielleicht, wo das Reh ist, damit ich ihm helfen kann. Es hat Probleme mit dem Herz und ich habe die richtigen Kräuter ihm zu helfen. Leidr kann man ja die Vögel nicht fragen, sondern nur hören, was sie sagen und wie ihr selber gemerkt habt, reden die Vögel meist nur Unsinniges, da habe ich das irgendwann aufgegeben.}

Die beiden Schwestern sind aufgeregt. Sie wissen natürlich, wo das weisse Reh ist. Im Wildgehege! Sie haben es ja selbst gestern erst gesehen! Die Hexe klatscht vor Freude in die Hände. 

\enquote{Wer hätte gedacht, dass ich es doch noch mit Hilfe der Erdbeeren finde!} Sie geht ins Haus und kommt mit einem Säckchen voller Kräuter zurück. 

\enquote{So ihr beiden. Seid so nett und helft mir, wir gehen gleich los.} Mit diesen Worten setellt sie einen alten Stuhl auf einen Bollerwagen mit einem Blick, der keinen Zweifel daran lässt, dass die beiden Schwestern sie ziehen sollen. Aber das ist für sie kein Problem. Es wird \textit{Das Wandern ist des Müllers Lust} und \textit{Hab mein Wagen voll geladen} gesungen und so vergeht die Zeit sehr schnell, bis sie beim Wildpark sind.

Die Hexe streckt die Hand aus und alle Rehe kommen angelaufen. Auch das weisse.

\enquote{Da bist du ja, haben sie dich hier eingesperrt, deswegen habe ich dich nicht gefunden.} sagt sie und hält ihm ihre Kräuter hin, die es auch frisst.

Sie begleiten die Hexe wieder nach Hause und versprechen sie nochmals besuchen zu kommen, so lange sie noch beim Opa sind. Am Abend gibt es schon wieder Bratwürste und rosa Erdebeermilch. Beide Schwestern sind erleichtert, dass die Vögel auch mit geschlossenen Augen wieder zwitschern.

\enquote{Ich glaube, dem weissen Reh geht es bald wieder besser} sagt 1.

Und Opa antwortet: \enquote{Wie kommst du denn jetzt darauf. Und wenn nicht, kommt es hier zu uns auf den Grill.} Opa muss zwar sehr lachen, die beiden Schwestern finden es aber gar nicht komisch. Dann doch etwas, dann immer mehr und zum Schluss müssen alle gemeinsam herzlich lachen.

 \hfill {\color{red}\decofourleft}

