\chapter*{\FontH{\Huge Die mutige Henna}}
\addcontentsline{toc}{chapter}{Die mutige Henna}
\lettrine[lines=3]{\color{red}D}{as} mutigste Mädchen, das ich kenne, heisst Henna. Sie ist meine beste Freundin. Mit ihr kann man immer tolle Sachen erleben. Wollt ihr die spannendste Geschichte hören, die ich je mit ihr erlebt habe? Also gut, ich erzähle sie Euch.

Alles begann damit, dass meien Pate Susanne mal wieder was mit mir unternehmen wollte. Susanne ist erst siebzehn Jahre alt und ziemlich cool.Keine meiner Freundinnen hat eine so junge Patentante. Von ihr weiss ich immer, welche Band gerade die beste ist und was es bedeutet, die Tage zu kriegen.

Sie hatte gerade furchtbaren Liebeskummer und wollte, wie sie sagte \enquote{deswegen} mit mir ein Schloss besichtigen. Was Liebeskummer mit einem Schloss zu tun haben könnte, weiss ich nicht, aber um ganz ehrlich zu sein, ich weiss gar nicht erst, was Liebeskummer bedeuten soll. Das macht aber offenbar nichts, denn wenn mir Susanne von ihrem Liebeskummer erzählt, presse ich die Lippen zusammen, gucke so wie Papa, wenn er von seiner Arbeit spricht und nicke dazu. Susanne findet, dass ich die einzige bin, die sie versteht.

Und eine Freundin solle ich mal auch mitnehmen, hat Susanne vorgeschlagen, dann werde es um so lustiger. Und damit hatte sie Recht. Bei so einer Einladung kam nur Henna in Frage. Da musste ich gar nicht lange überlegen. Wir drei vereinbarten, uns gegen späten Nachmittag beim Schloss zu treffen.

Schon gleich nach der Kasse begannen unsere Probleme, ohne das Henna und ich das aber wissen konnten. Ein junger Mann im Alter von Henna stand hinter uns und irgendwie sind Susanne und er ins Gespräch gekommen. Die ersten beiden Räume, die wir im Schloss besichtigten, stand er die ganze Zeit neben uns, dann entschuldigten sich er und Susanne, siw würden gerne einen Kaffee im Museumscaf'e trinken gehen wollen. Wir genehmigten gerne und machten uns auf den Weg durchs Schloss.

Das Schloss war wunderschön und könnte auch genauso in einem Märchenbuch gemalt worden sein. Hohe spitze Türme an allen Seiten, ringsherum ein Park, der mit einer hohen Mauer umgeben war. Wie bei einer Burg. Die Zimmer waren laut Tafeln so, wie sie früher dann und dann gemäss der Phantasie des Schilderschreibers gewesen sein müssen. In jedem Zimmer ein Kachelofen und dann überall Rüstungen, Waffen und Besteck von damals. Wenn ich so ein Schloss gehabt hätte, hätte ich die Sachen wohl anders aufbewahrt, aber die Menschen früher haben sich wohl noch nicht so viele Gedanken gemacht wie wir jetzt. Das merke man schon daran, was sie für ein WC gehalten haben, meinte jedenfalls Henna, und da hatte sie Recht.

Wir sind in jedem Raum gewesen und haben uns oft über die Menschen damals lustig gemacht. Henna meinte, dass die früher hier gelebt haben, noch an Zauberei, Drachen und Gespenster geglaubt haben. Wie sie das gerade sagt, hören wir ein lautes Donnern von draussen.

Eigenartigerweise ist in dem Schloss niemand zu sehen, als wir raus laufen, um nach zu sehen. Auch im Park ist keine Menschenseele, dabei waren es noch so viele, als wir gekommen waren. Das grosse Tor durch die Mauer ist zu. Eine schwere Holztür ist da, wo wir reingekommen sind. Aus Laibeskräften fangen wir sofort an zu rufen. Aber niemand hört.

Henna unterbricht mich. Es habe keinen Sinn so zu schreien. Auf ihren Rat hin laufen wir der Mauer entlang, um nachzusehen, ob wir nicht sonst irgendwo raus kommen, denn raus wollen wir hier. Das ist sicher. So eine Nacht in einem Schloss verbringen?

Aber da ist kein anderes Tor und keine Lücke. Die Mauer ist dicht, da kommt niemand rein, dass haben die damals schon richtig gemacht, aber es kommt eben auch niemand raus. Henna meint, in jedem Schloss gebe es geheime Fluchtwege. Man müsse nur in der Bibliothek an einem bestimmten Bich ziehen, dann drehe sich die Wand und man kommt in einen unterirdischen Tunnel. Das haben alle Schlösser, das kenne sie aus dem Fernsehen.

Daher sind wir wieder in das Schloss rein. Es wurde eh dunkel. Wir haben in allen Zimmern nach der Bibliothek gescuht, aber keien gefunden. Bücher gibt es im ganzen Schloss nicht. Wir kommen noich gar nicht dazu uns Gedanken zu machen, wie wir jetzt weiter machen, nachdem wir keine Bibliothek gefunden haben, denn wir werden von einem lauten Schlag erschreckt. Ein Dinnern, der durch das ganze Schloss dringt.

Ich bleibe wie angewurzelt stehen, aber Henna meint, das sein nur die Eingangstüre, die hätten wir offen gelassen und ein Windstoss habe sie zugeschlagen.

Da hat sie schon wieder Recht, denn es ist auch ganz still geworden. Mit geschlossener Tür ist nichts mehr zu hören, ausser wie Henna atmet. Sie ist Allergikerin, wahrscheinlich sind irgendwelche Pollen gerade in der Luft. Noch schlimmer ist es bei Katen. Katzenhaare merkt Henna sofort. Wenn ich die katze von Susanne streichekle und dann Hennna besuche, muss sie sofort husten.

Es ist nicht nur still, sondern mit der zeit auch dunkel geworden, denn sie Sonne ging nun unter. Ob Susanne überhaupt bemerkt hat, dass wir weg sind? Henna und ich versuchen einen Lichtschalter zu finden. Aber entweder sind die sehr gut versteckt, oder hier machen die noch Licht mit Kerzen. Jedenfalls suchen wir neben jeder Tür nach einem Schalter, zu Hause bei mir und bei Henna und bei jedem, den wir kennen, sind alle Lichtschalter neben einer Tür.

Ich überlege gerade, wie das bei Onkel Tonis Stall ist, der ist auch ziemlich gross, so wie das Schloss und vielleicht ist das mit den Lichtschaltern ja anders in grossen Gebäuden. In dem Augenblick wir wir ein lesies Tapsen einen Stocj über uns. Ic halte vor Schreck die Luft an. Auch Henna lauscht gespannt.

Wir hören zwar nichts mehr, aber henna ruft trotzdem ganz laut, ob da jemand sein. Ich bin sher erschrocken von der Lautstärke und von dem Echo. Nioemand antwortet. Henna ruft nochmals. Nichts. Doch dann war wieder das Tapsen zu hören. Henna läuft sofort los um nachzusehen. Ich habe viel zu viel Angst, hier alleine stehen zu bleiben, also folge ich ihr. Die Treppe hoch, dort kam das Geräusch her. Mir bleibt fast das Herz stehen. Das Tapsen ist gabz klar aus dem Zimmer am Ende des Ganges zu hören. Der Mond scheint sehr hell, es ist zwar alles zu erkennen, aber nur schwarzweiss. Henna geht zielstrebig auf das Zimmer zu. Mein Herz schlägt so laut, als ob irgendjemand ganz laut Musik spielt. Ich bin sicher, dass mindestens Henna das hören kann. 

Wir nähern uns dem Zimmer. Ich weiss nicht, warum genau wir eigentlich nachsehen wollten und ob es nicht besser gewesen wäre, sich zu verstecken und das wollte ich Henna aucg gerade sagen, da ist sie schon in dem Zimmer und um die Ecke. Ich schnell hinterher. Das Tapsen kam vom anderen Ende des Zimmers. Der Mond malt einen langen Schatten auf den Boden.

Plötzlich wird dieser Schatten länger und immer länger und so lang, dass er erst bis zur Wand hinter unserem Rücken reicht und dann ist ein lauter Schlag zu hören, viel lauter und heller las der vorhin und tausende Scherben fliegen uns entgegen. Irgendetwas ist kaputt gegangen. So etwas wie eine riesige Tasse oder so, jedenfalls sehen die Scherben so aus.

Ich traute mich nicht einmal mit den Wimpern zu schlagen. Henna springt aber nach vorne, greift zu, ein Fauchen ist zu hören. Ich konnte das alles nicht so richtig erkennen, aber ich höre henna rufen \enquote{Übel, ganz übel \dots eine Katze!}

Und tatsächlich hält sie eine Katze im Arm. Wie ihr euch vorstellen könnt, bin ich noch nie im Leben so erleichtert gewesen wie da. Nie wäre ic hauf eine Katze gekommen! Ich dachte, das sei ein Gespenst oder mindestens ein Vampir oder ganz und gar ein Einbrecher, aber icher keine Katze! henna sah das anders. Sofort fing sie an zu husten. Ich nahm ihr die Katze sofort ab und scheuchte sie raus. Die muss wohl reingekommen sein, als wir die Tür aufgelassen hattem dachte ich so.

Henna fand ein ein richtiges WC, so ein eins für Besucher, da hat sie sich die Hände gründlich gewaschen. Dann ging es mi dem Husten auch gleich besser. Dann haben wir noch geflucht, nichts zum Essen zu haben, nur Wasser aus der Leitung gab es hier. Das Bett der Königin hat uns dann aber wieder versöhnt. Riesig und ein Himmelbett noch dazu. Die Decke goldbestickt, aber sehr muffig.

So haben wir Prinzessinnen geschlafen, bis uns die ersten Besucher am nächsten Tag gweckt hatten. Mit uns haben die nicht gerechnet. Angst hatte in der Nacht übrigens keine von uns mehr, nicht einmal ich!




