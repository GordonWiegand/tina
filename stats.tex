\chapter*{\large\textsf{Statistiken und harmonischer Zipfel}}
        \thispagestyle{empty}

Bei einem Buch handelt es sich immer auch um einen linguistischen Korpus. Schon
das Wort \emph{Korpus} deutet an, dass durch diese Umbenennung aus einem
lebendigen Text voller Feen und Elfen eine Tote Liste aller enthaltener Wörter
gemacht wurde. Und solche Listen ziehen Statistiker an, die dann anfangen
Erbsen und alle anderen Wörter zu zählen.

Wie so oft bei Wissenschaft ist das Ergebnis erwartbar. Auf den ersten Plätzen
rangieren:

\begin{enumerate}
    \item die (1'021)
    \item und (989)
    \item sie (661)
    \item der (582)
    \item das (476)
    \item zu (415)
    \item nicht (387)
    \item es (363)
    \item aber (350)
    \item er (350)
\end{enumerate}

Wenn man Menschen glaubt, die solch absurdes Zählen zu ihrem Beruf gewählt
haben, ist es nicht Schreibfaulheit, die so viele kurze Wörter in die Charts
gebracht hat. Nein, das ist so üblich. Wie Ameisen: klein, aber viele.

Auch Wörter sind wie Gesellschaften. Es gibt die graue Masse an Arbeitstieren, die
Liste liefert Beispiele, die, über die niemand reden will und Wörter mit
Migrationshintergrund. Die Diskussionen sind da ganz ähnlich wie bei
menschlichen Gesellschaften auch. Alles Fremde wird abgelehnt und erst einmal
nicht zugelassen und zwei Generationen später mag sich niemand mehr erinner,
dass das Wort Grosseltern hat,  die ganz wo anders geboren wurden. Und es gibt
Popstars unter den Wörtern. 

Per Geburt haben bei Wörtern nur Substantive das
Zeug zu Popstars. Das erste Substantiv in diesem Buch ist auf Platz 71 die \emph{Maus}, dann auf Platz 88
\emph{Menschen} und genau auf Platz 100 \emph{Mama}. Das grosse \emph{M} hat
unter allen grossen Anfangsbuchstaben
klar das Rennen für sich entschieden. Konkurrenzlos drei Mal in den Top\,100.
Die oben genannte \emph{Fee} landet auf Platz 500 und auf Platz 1'000 ist
\emph{Schönheit}. Na ja, es steht an tausendster Stelle in der Liste -- mit zwei
Nennungen gibt es aber noch sehr viele andere Wörter auf demselben Platz, der
tatsächlich der 112te ist.

Im Grunde gibt es aus der quantitativen Linguistik nur eine mir bekannte
interessante Erkenntnis. Die Verteilung der Ranghäufigkeit (wie oft kommt das
häufigste Wort vor, wie häufig das Zweithäufigste usw.). Sie folgt einer Verteilung,
die nach ihrem Entdecker Zipf benannt ist. Der Name klingt ein wenige wie eine
Figur aus einem Buch von Wilhelm Busch. Und diese Zipf'sche Verteilung folgt einer Reihe mit noch mehr Wohlklang,
der sogenannten harmonischen Reihe
$\left(\frac{1}{1},\frac{1}{2},\frac{1}{3},\dots\right)$. In Worten: eins
geteilt durch Rang ergibt Anteil. So ungefähr jedenfalls, wer es genauer wissen
möchte, bemühe bitte Wikipedia.

Es gibt jetzt also die theoretische Verteilung der Wörter nach dem harmonischen
Zipf und die tatsächliche Verteilung
in diesem Buch. Plotten wir beide zusammen ergibt sich eine Grafik wie in unten
stehender
Abbildung. Ein Statistiker sagt dann, die Übereinstimmung ist -- man
entschuldige den sehr technischen, wissenschaftlichen Ausdruck -- \emph{ganz
okay}.

\begin{figure}[ht]
\centering
	\includegraphics[height=.6\textheight,width=\textwidth]{tinaStat/zipfAlle.pdf}
	\label{dend1}
\end{figure}
