\chapter*{\FontH{\Huge Aus dem Leben des Karl Rebosam}}
\addcontentsline{toc}{chapter}{Aus dem Leben des Karl Rebosam}
Karl ist auch in den Grundfragen der Pädagigik unorthodox. Je nach Vergehen des Zöglings schlägt sich Karl selbst ins Gesicht. Blut ist ein \emph{must}. Das soll dem Knaben eine Lehre sein!

Gelegentlich ertappt sich Karl bei dem Gedanken, sein Kind zur Adoption frei zu geben.
\begin{center}
{\huge \textthing}
\end{center}
Karl ist heute sehr nachdenklich. Er hatte folgenden Algorithmus angewendet:
\begin{enumerate}
	\item Öffne das Telefonbuch.
	\item Bilde die Quersumme der Telefonnummer des ersten Eintrags beim Buchstaben \emph{A}.
	\item Notiere das Ergebnis und verfahre so bei allen weiteren Buchstaben.
	\item Addiere alle Ergebnisse und bilde die Quersumme.
	\item Bilde aus diesem Wert so lange die Quersumme, bis diese einstellig ist.
	\item Wiederhole alle vorhergehenden Schritte für den jeweils letzten Eintrag eines Buchstabens.
\end{enumerate}
Das Ergebnis erwies sich in beiden Fällen als $5$. Was hat das zu bedeuten? Voll innerlicher Welkeheit schritt Karl seine Sammlung ausgestopfter Nagetiere ab und empfand es einmal mehr als schmerzlich, dass ausgerechnet eine Ratte noch nicht zu integrieren gewesen war.
\begin{center}
{\huge \textthing}
\end{center}
An jedem achten Tag nach Neumond empfand Karl Rebosam Lust. Schon der Anblick der morgentlichen Zahnpasta, wie sie sich schlangenfraugleich aus der Tube auf seine Bürste windet und er diese dann in seinen Mund schiebt und nach einem kurzen Augenblick des Innhaltens sanft aber kräftig mit der Zunge zerdrückt, liess heissen Schweiss aus seiner Stirn triefen. 

Und wenn dann der  Toaster durch die Behandlung mit feuriger Glut aus zwei bleichen schwammigen Fladen, braune knusprige Wesen erschafft und diese mit einem Knall aus seinem inneren heraus spuckt, ist Karl der Ohnmacht jeweils schon recht nahe.
\begin{center}
{\huge \textthing}
\end{center}
Es stellte sich als kapitalen Fehler heraus, das falsch gegebene Rückgeld zu reklamieren. Die Bäckersfrau war nicht bereit, Kritik zu akzeptieren. Diese kategorische Charaktereigenschaft, gepaart mit einer Meisterschaft in der Schwitzkastentechnik, führten zu einer für Karl nicht vorhersehbaren Reaktion.

Karl verbrachte einen wesentlichen Teil des Vormittags unter der Achselhöle der sonst frommen Bäckersfrau. Er musste mit Respekt beobachten, wie die Bedienung der ihm nachfolgenden Kundschaft ganz reibungslos durch die Vermittlung nur eines Arms funktionierte. Im Grunde klappe sie sogar besser als mir beiden Armen, denn da Katls Gesicht beim Bezahlen jeweils tief in die Kasse gedrückt wurde, konnte er bei der Herausgabe des Wechselgeldes mitzählen. Ein weiterer Fehler war dabei der Bäckersfrau nicht vorzuwerfen. Ärgerlich war im Weiteren nur, dass er seine Pfeife nicht mitgenommen hatte.
\begin{center}
{\huge \textthing}
\end{center}
Heute hat Karl folgende annonce im Lokalblatt publizieren lassen: {\itshape Hundesalon Rebosam sucht per sofort einen Hundecoiffeur. Neben viel Libe zu den Vierbeinern erwarten wir einen männlichen Berwerber, der aus Gründen der Authenzität möglichst selbst stark behaart ist. Ausnahmen gelten nur für Träger eines Menjou oder eines Dali-Bärtchens. Weitere Auskunft erhalten sie unter Tel. \ldots. }

Natürlich gibt es diesen Hundesalon Rebosam gar nicht. Die Telefonnummer ist aber tatsächlich die unseres Karln.
\begin{center}
{\huge \textthing}
\end{center}
=======
Schosmiroslaw Brauchtitsch, der Grossvater von Karl Rebosam lernte erst als Siebzigjähriger das Lesen. Seine einzige Lektüre in seinem Leben blieb ein Abenteuerroman. Völlig gefesselt von der Handlung ignorierte er den um ihn herum krachenden Krieg, einem von jenen Kriegen, die heutzutage ob ihrer damaligen Inflation nicht einmal mehr sogenannte Experten und Eingeweihte kennen. 

Als jedenfalls Schosmiroslaws Dorf von den feindlichen Truppen überrannt wurde, wurde auch sein Haus inspiziert. Völlig unter dem Einfluss des Grossenromans stehend, war Schosmiroslaw inenrlich zum Held gereift und warf dem eintretenden Offizier in Ermangelung eines Fehdehandschuhs ein Holzscheid ins Gesicht. Selbstverständlich wirde er stante pede erschossen, was ihm Gelegenheit gab so zu sterben, wie er es sich erst seit jüngstem erträumt hatte.
\begin{center}
{\huge \textthing}
\end{center}
Ganz überraschend hat es heute an der Tür von Herrn Rebosam sehr heftig geklopft. Unwillig zu öffnen hat Herr Rebosam eine Weile zurück geklopft. Viel zu spät erst hat er sich gefragt, ob es sich vielleicht nicht doch gelohnt hätte, zu öffnen.
\hfill {\color{red}\decofourleft}



Herr Maestro. Erlauben sie mir bitte, mit einer sehr persönlichen Bitte zu beginnen: erzählen sie uns, von ihren Träumen.

Da ich stark kurzsichtig bin und im Allgemeinen vor dem zu Bett gehen die Brille abnehme, könnte man meinen, dass ich sehr verschwommen träume. Das ist allerdings nicht der Fall, da ich im Traume sehr wohl meine Brille trage. Ich träume also ähnlich scharf wie sie auch. Dabei fällt mir ein, dass mir schon als Kind nach dem Verzehr einer Mon Cherry Packung aufgefallen ist, ist der Umstand auch im Traume alkoholisiert zu sein, wenn ich das vor dem zu Bett gehen tatsächlich gewesen bin. Ich frage mich, ob dies spezifisch an meiner Person liegt, oder es sich um eine universelle Eigenschaft des Träumens handelt.



Lassen sie uns über ihre Arbeit reden. Sie leiten den grössten Laienchor Deutschlands.
