\chapter*{\FontH{\Huge Der See im Dschungel}}
\addcontentsline{toc}{chapter}{Der See im Dschungel}
\lettrine[lines=3]{\color{red}T}{ief} mitten im grössten Dschungel von Indien liegt ein See. Die Menschen haben keinen Namen für ihn, denn nur ganz wenige haben ihn bisher gesehen. Einer der wenigen, die es schon so tief in den Dschungel geschafft hatten, war der Jäger Amar. Von ihm weiss ich diese Geschichte, er hat sie selbst beobachtet.

Der See war weit und breit der einzige. Menschen müssen mindestens vier bis fünf Tage marscheiren, um den nächsten zu erreichen, so weit weg ist der. Für die Tiere war der damit der einzige Ort, an 
dem Sie Baden konnten, was wichtig ist für Dschungeltiere, wenn sie beispielsweise einmal versehentlich in einem Ameisenhaufen gestanden hatten und die nun an ihnen hochkrabbelten und bissen und zwackten. Und auch für Tiere gilt, dass sie gerne baden, wenn es heiss ist.

Hirsche, Wildschweine, Muntjaks, Antilopen und noch viele andere Tiere wollten daher zu diesem See, um da zu baden. Bei diesem See lebte aber ein riesiger Tiger, den grössten, den Amar je gesehen hatte. Und dieser Tiger erlaubte nicht, dass andere ausser ihm in diesem See baden durften. Sobald sich ein Tier unvorsichtig dem See näherte, 


 \hfill {\color{red}\decofourleft}
