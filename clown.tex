\chapter*{\FontH{\Huge Der grosse Clown kann fliegen}}
\addcontentsline{toc}{chapter}{Der grosse Clown kann fliegen}
\lettrine[lines=3]{\color{red}L}{ouis} und seine Eltern kommen gerade vom Stadtfest zurück. Er hatte im Karussell fahren dürfen und hatte einen Schleckstengel bekommen, der viel grösser ist als seine Hand. Aber sein ganzer Stolz gehört einem Luftballon, der aussieht wie ein Clown. Louis hält den Ballon auf dem Arm.

Als Mama die gerade die Tür aufschliesst kommt die Nachbarin Frau Schweizer.

\enquote{Hach Frau Reichenbach}, ruft sie, \enquote{Sie können es nicht glauben. Meine Neffin liegt im Krankenhaus, ein Unfall, wissen Sie. Die sollte ich dringend besuchen. Ob sie nicht so lange auf Leika aufpassen könnten? Heute Abend bin ich bestimmt zurück.}

Leika ist der Hund von Frau Schweizer. Ein sehr kleiner weisser Hund, der immer kläfft. Louis kann Leika deswegen nicht besonders leiden. Nie kann er mit dem Ball draussen spielen, wenn Leika in der Nähe ist. Der Hund springt dann immer hinter dem Ball her und versuchte den zu beissen, was aber nie klappt, denn sein Maul ist viel zu klein. Frau Schweizer ruft dann immer, dass Leika nur spielen will, womit Louis aber auch nicht geholfen ist. Gegen den Ball treten geht dann nicht mehr ohne Leika weh zu tun.

Aber einen Nachmittag lang kann ja nicht so schlimm sein, denkt sich Louis. Was für ein Irrtum! Schon als allererstes schnappt sich Leika den linken Hausschuh von Louis und rennt damit auf den Balkon.

\enquote{Zieh endlich deine Hausschuhe an!} hört Louis Papa rufen. Pff, als ob er das nicht vorgehabt hätte. Dann schmeisst Leika den Turm um, den Louis und Papa heute Morgen, als Mama noch geschlafen hat, gebaut haben. Louis fängt an mit Leika zu schimpfen, aber Mama ruft aus der Küche, dass das doch nicht so schlimm sein und man den Turm neu bauen könne. Aber Louis will keinen neuen Turm, er will den Turm wieder haben. Er streckt Leika die Zunge raus und schiesst ein Kissen nach ihm.

\enquote{Jetzt reicht es aber!} Papa und Mama sind ärgerlich. Mit Kissen nach kleinen Hunden schiessen ist nicht in Ordnung, finden sie. Als ob dem das weh tun könnte.

\enquote{Und ab in dein Zimmer, Freundchen, bis es Mittag gibt!} heisst es dann auch noch. So eine Ungerechtigkeit, aber egal, Louis will sowieso in sein Zimmer und etwas spielen. Tür zu und erst wieder raus kommen, wenn Frau Schweizer Leika wieder abhgeholt hat, so ist sein Plan.

Aber bevor er die Tür schliessen kann, wird Leika auch in sein Zimmer geschoben. 

\enquote{Du spielst jetzt auch mit Leika}, sagt Mama, \enquote{Wir wollen hier noch im Wohnzimmer putzen und da stört ein Hund sehr. Spiel du mit ihm.}

Das hatte noch gefehlt! Jetzt muss sich Louis um Leika kümmern. Mist! Aber vielleicht genügt es ja, ihn zu ignorieren. Louis sieht sich gerade um, als Leika furchtbar anfängt zu bellen und sich auf den neuen Ballon stürzt. Ausgerechnet! Gerade noch kann Lois ihn wegreissen. Das war knapp. Einen Ball kann Leika nicht zerbeissen, bei einem Ballon ist sich Louis da nicht so sicher.

Er hält den Clown so hoch er kann, aber Leika springt immer wieder an ihm hoch. So ist der Clownsballon zwar erst einmal in Sicherheit, aber lange kann er die Arme nicht so nach oben halten. Dann eben in das Regal mit dem Ballon. Das erste Fach, bei dem es Louis probiert ist schon voll. Kuscheltiere. Das zweite ist zu klein. Beim dritten klappt es. Triumphieren schiebt Louis den Ballon in das Fach.

Leika kläfft dazu die ganze Zeit ununterbrochen. Er will unbedingt an diesen Ballon kommen. So sind Hunde halt, das ist ihre Art zu spielen, aber es nervt eben doch. Plötzlich ist Leika mit einem Sprung, von dem Louis nie gedacht hätte, dass ein so kleiner Hund das kann, auf dem Schreibtisch. Und von dort ist es nicht mehr weit bis zum Regal. 

Louis kann im letzten Moment den Clownsballon aus dem Regal reissen. Dabei fällt er ihm aber durch das offene Fenster.

Selbst Leika verschlägt es den Atem. Louis und Leika sind still und sehen zum offenen Fenster und zu dem Ballon. Louis überlegt gerade, ob es wohl laut poltert, wenn so ein Clownsballon unten auf dem Boden aufschlägt. Und ob er dann wohl in tausend Scherben zerspringen wird, wie neulich Papas Glas, oder ob er nur eine grosse Beule bekommt, wie sein Spielzeugauto.

Aber es passiert etwas ganz anderes, etwas, womit weder Louis noch Leika gerechnet haben. Der Ballon steigt in die Luft!

\enquote{Der grosse Clown kann fliegen!} ruft Louis immer wieder ganz aufgeregt.

Mama und Papa kommen ins Zimmer und wollen wissen, was hier los ist.

\enquote{Hast du nicht aufgepasst?} fragt Papa. \enquote{Na der ist weg.} ergänzt Mama. Beide wollen Louis trösten, aber das ist gar nicht nötig! Der Clown wird schon wissen, warum er weg geflogen ist. Und Louis weiss es auch. Jetzt kann Leika ihm nichts mehr anhaben, jetzt ist er sicher.

Und weil Louis das weiss, macht es ihm auch gar nichts mehr, sich den Rest des nachmittags ein bisschen um den Hund zu kümmern.

 \hfill {\color{red}\decofourleft}
